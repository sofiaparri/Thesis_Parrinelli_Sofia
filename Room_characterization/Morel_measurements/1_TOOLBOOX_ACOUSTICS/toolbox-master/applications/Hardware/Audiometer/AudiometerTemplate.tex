% platzhalten <\nameDesPlatzhalters> wird von matlab ersetzt

\newcommand{\itaBriefkopfBild}{<\itaBriefkopfBild>}
\newcommand{\datumDerMessung}{<\datumDerMessung>}
% \newcommand{\nameDesPruefers}{<\nameDesPruefers>}
\newcommand{\proband}{<\proband>}
\newcommand{\kommentar}{<\kommentar>}
\newcommand{\dateinameGrafik}{<\dateinameGrafik>}
\newcommand{\geburtsDatum}{<\geburtsDatum>}


\documentclass[11pt,a4paper, twoside, final]{scrartcl}
\usepackage{graphicx}
\usepackage[latin1]{inputenc}
\usepackage{lscape}

\pagestyle{plain}

\setlength{\textwidth}{17.5cm}
\setlength{\oddsidemargin}{-1cm}
\setlength{\evensidemargin}{1cm}
\setlength{\topmargin}{-2cm}
\setlength{\headheight}{0cm}
\setlength{\headsep}{0cm}
\setlength{\footskip}{0cm}
\setlength{\textheight}{27.5cm}

\pagestyle{empty}


\begin{document}

\parindent 0pt 

 \includegraphics[width=1.0\textwidth]{\itaBriefkopfBild}

\begin{center}
		\huge
		Messprotokoll zur Messung der H�rkurve\\
		nach DIN EN ISO 8253-1
\end{center}

\begin{tabular}{|p{0.43\textwidth}p{0.51\textwidth}|}\hline
	\multicolumn{2}{|p{0.99\textwidth}|}{
	Bitte beachten Sie, dass dieses Audiogramm nicht die Diagnose eines Arztes ersetzt. Es kann jedoch ein erster Hinweis auf eine Beeintr�chtigung des H�rens sein.}\\ 
	&\\
		Pr�fdatum: & \datumDerMessung \\
  	 Name: & \proband\\ 
  	   	 Geburtsdatum: & \geburtsDatum \\ 
		 Kommentar: & \kommentar \\
		\hline
\end{tabular}


\begin{figure}[htbp]
\centering
		\includegraphics[width=1\textwidth]{\dateinameGrafik}
	\caption{Audiometrie von \proband}
	\label{fig:ZweiProben_gemittelt}
\end{figure}



\end{document}