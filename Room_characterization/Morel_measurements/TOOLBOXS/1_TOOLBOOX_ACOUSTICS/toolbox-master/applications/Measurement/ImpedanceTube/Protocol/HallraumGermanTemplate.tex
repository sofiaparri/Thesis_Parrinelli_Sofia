% platzhalten <nameDesPlatzhalters> wird von matlab ersetzt




\newcommand{\itaBriefkopfBild}{KopfzeileGKB.png}
%\newcommand{\datumDerMessung}{22.11.2111}
\newcommand{\nameDesPruefers}{Martin Guski}
%\newcommand{\temperaturInGradC}{21,5}
%\newcommand{\luftfeuchtigkeit}{52,3}\newcommand{\nameDerProbe}{NMC A}
%\newcommand{\anzahlDerMessungen}{11}
%\newcommand{\samplesOderWiederholungen}{Samples}
%\newcommand{\dateinameGrafik}{alpha_aixFOAMDirektAufBoden.pdf}
%\newcommand{\beschreibungDesMaterials}{	Anthrazitfarbiger, teils geschlossen-poriger Zellschaum. Dicke ca. 52 mm
%	Passgenaue Platzierung der Samples im Probenhalter mit dichtem Abschluss zur Wandung. 
%	Einbau der Proben ohne Abstand vor dem schallharten Rohrabschluss.}



\documentclass[11pt,a4paper, twoside, final]{scrartcl}
\usepackage{graphicx}
\usepackage[latin1]{inputenc}
\usepackage{lscape}

\pagestyle{plain}

\setlength{\textwidth}{17.5cm}
\setlength{\oddsidemargin}{-1cm}
\setlength{\evensidemargin}{1cm}
\setlength{\topmargin}{-2cm}
\setlength{\headheight}{0cm}
\setlength{\headsep}{0cm}
\setlength{\footskip}{0cm}
\setlength{\textheight}{27.5cm}

\pagestyle{empty}



\begin{document}

\parindent 0pt 

 \includegraphics[width=1.0\textwidth]{\itaBriefkopfBild}

\begin{center}
		\huge
		Messung der Schallabsorption in Hallr�umen nach ISO 354
\end{center}

\begin{tabular}{p{0.43\textwidth}p{0.51\textwidth}}
	\multicolumn{2}{p{0.99\textwidth}}{
			\underline{Beschreibung der Pr�feinrichtung:} \newline
			Hallraum des Instituts f�r Technische Akusik der RWTH-Aachen  
	}\\ 
	
	
	&\\
		Hallraum Volumen:  		& 122.6 $m^3$ 						\\
		Hallraum Oberfl�che: 	& 177.6 $m^2$	            \\
		Temperatur: 					&	<temperaturInGradC> ~�C \\
		Luftfeuchtigkeit: 		& <luftfeuchtigkeit>  ~\% \\ 
		Pr�fdatum: 						& <datumDerMessung>       \\
		Pr�fer: 							& <nameDesPruefers>       \\ 
	& \\
		 		 
	\multicolumn{2}{p{0.99\textwidth}}{	
		\underline{Beschreibung des zu untersuchenden Materials und der Pr�fanordnung:} 
		} \\
		
	Bezeichnung der untersuchten Probe: & <NameDerProbe> \\  
	 & \\
	 
	\multicolumn{2}{p{0.99\textwidth}}{	
			<beschreibungDesMaterials> 
	}\\
	
	
	
\end{tabular}



% PLOT
\begin{figure}[htbp]
\centering
		\includegraphics[width=1\textwidth]{<Dateiname der Grafik>}
	\caption{<Bildunterschrift>}
\end{figure}



% TABELLE 
\small
\centering
\begin{tabular}{lcccccccccccccccccccc}
 $f / [Hz]$   & 100 &  125  &  160  &  200  &  250  &  315  &  400  &  500  &  630  &  800  \\  \hline 
 $\alpha_S$     & <alpha_s_100>  & <alpha_s_125>  &  <alpha_s_160> &  <alpha_s_200> &  <alpha_s_250> &  <alpha_s_315> &  <alpha_s_400> &  <alpha_s_500> &  <alpha_s_630> & <alpha_s_800> \\ \\
 $f / [Hz]$ &  1000 &  1250 &  1600 &  2000 &  2500 &  3150 &  4000 &  5000 &  6300  & 8000 \\ \hline
 $\alpha_S$   &  <alpha_s_1000> &  <alpha_s_1250>  &  <alpha_s_1600>  & <alpha_s_2000> &  <alpha_s_2500> &  <alpha_s_3150> &  <alpha_s_4000> &  <alpha_s_5000> &  <alpha_s_6300> & <alpha_s_8000>   \\ 
  
\end{tabular}
\\

\end{document}