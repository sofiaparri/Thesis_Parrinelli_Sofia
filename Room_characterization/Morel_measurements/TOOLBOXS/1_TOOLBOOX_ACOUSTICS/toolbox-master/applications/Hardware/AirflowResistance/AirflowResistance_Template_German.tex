%% platzhalten <\nameDesPlatzhalters> wird von matlab ersetzt
%
\newcommand{\itaBriefkopfBild}{<\itaBriefkopfBild>}
\newcommand{\datumDerMessung}{<\datumDerMessung>}
\newcommand{\nameDesPruefers}{<\nameDesPruefers>}
\newcommand{\nameDerProbe}{<\nameDerProbe>}
\newcommand{\anzahlDerMessungen}{<\anzahlDerMessungen>}
\newcommand{\beschreibungDesMaterials}{<\beschreibungDesMaterials>}
\newcommand{\materialDicke}{<\materialDicke>}
\newcommand{\materialGewicht}{<\materialGewicht>}
\newcommand{\materialDichte}{<\materialDichte>}

\newcommand{\strWidMittelwert}{<\strWidMittelwert>}
\newcommand{\strWidStandardabweichung}{<\strWidStandardabweichung>}
\newcommand{\spezStrWidMittelwert}{<\spezStrWidMittelwert>}
\newcommand{\spezStrWidStandardabweichung}{<\spezStrWidStandardabweichung>}
\newcommand{\laengenbezStrWidMittelwert}{<\laengenbezStrWidMittelwert>}
\newcommand{\laengenbezStrWidStandardabweichung}{<\laengenbezStrWidStandardabweichung>}





\newcommand{\itaBriefkopfBild}{KopfzeileGKB.png}
\newcommand{\datumDerMessung}{03.01.2011}
\newcommand{\nameDesPruefers}{Martin Guski}
\newcommand{\nameDerProbe}{FlowTec Red}
\newcommand{\anzahlDerMessungen}{9}
\newcommand{\beschreibungDesMaterials}{	Die Probe wurde  so und so befestigt...}
\newcommand{\materialDicke}{15 mm }
\newcommand{\materialGewicht}{ 35 g}
\newcommand{\materialDichte}{100 kg/m $^3$}

\newcommand{\strWidMittelwert}{ 110 Pa s/m$^3$}
\newcommand{\strWidStandardabweichung}{7}
\newcommand{\spezStrWidMittelwert}{111 Pa s/m }
\newcommand{\spezStrWidStandardabweichung}{12}
\newcommand{\laengenbezStrWidMittelwert}{7.5 kPa s/m$^2$}
\newcommand{\laengenbezStrWidStandardabweichung}{13}












\documentclass[11pt,a4paper, twoside, final]{scrartcl}
\usepackage{graphicx}
\usepackage[latin1]{inputenc}
\usepackage{lscape}

\pagestyle{plain}

\setlength{\textwidth}{17.5cm}
\setlength{\oddsidemargin}{-1cm}
\setlength{\evensidemargin}{1cm}
\setlength{\topmargin}{-2cm}
\setlength{\headheight}{0cm}
\setlength{\headsep}{0cm}
\setlength{\footskip}{0cm}
\setlength{\textheight}{27.5cm}

\pagestyle{empty}



\begin{document}

\parindent 0pt 

 \includegraphics[width=1.0\textwidth]{\itaBriefkopfBild}

\begin{center}
		\huge
		Messprotokoll zur Messung des Str�mungswiderstandes nach DIN EN 29053
\end{center}


\begin{tabular}{|p{0.43\textwidth}p{0.51\textwidth}|}\hline

%%%%%%%%%%%%%%%%%%%%%%%%%%%%%%%%%%%%%%%%%%%%%%%%%%%%%%%%%%%%%%%%%%%5
% PRUEFEINRICHTUNG
	\multicolumn{2}{|p{0.99\textwidth}|}{
	\underline{Beschreibung der Pr�feinrichtung:} \newline
		Die Messung des Str�mungswiderstandes erfolgte gem�� DIN EN 29053 mit dem Luftwechselstromverfahren (Verfahren B) . Das zylindrische Pr�fgef�� hat einen Durchmesser von ??? mm  und eine H�he von ??? mm. }\\ 
	&\\
	
					Pr�fdatum: & \datumDerMessung \\
		 Pr�fer: & \nameDesPruefers\\ 
	&\\
	\hline \multicolumn{2}{c}{ }\\ 

%%%%%%%%%%%%%%%%%%%%%%%%%%%%%%%%%%%%%%%%%%%%%%%%%%%%%%%%%%%%%%%%%%%5
% MATERIAL
	\hline
	\multicolumn{2}{|p{0.99\textwidth}|}{	
	\underline{Beschreibung des zu untersuchenden Materials und der Pr�fanordnung:}\newline
	\beschreibungDesMaterials }\\
	 & \\
			Bezeichnung der untersuchten Probe: & \nameDerProbe \\ 
			Dicke der Probe: & \materialDicke \\
			Dichte der Probe: & \materialDichte \\
			
			Anzahl der Messungen: & \anzahlDerMessungen \\
	
	&\\
	\hline
	\multicolumn{2}{c}{ }\\ 
	


	
	
	
\end{tabular}



\begin{tabular}{|p{0.43\textwidth}p{0.21\textwidth}p{0.3\textwidth}|}
%%%%%%%%%%%%%%%%%%%%%%%%%%%%%%%%%%%%%%%%%%%%%%%%%%%%%%%%%%%%%%%%%%%5
% ERGEBNISSE
	\hline
	\multicolumn{3}{|p{0.99\textwidth}|}{
			\underline{Ergebnisse der Messungen} \newline
			Messung erfolgte einer linearen Str�mungsgeschindigkeit von  u = bla, interp auf 0,5 mm/s 
			}\\ 
	&&\\
		Str�mungswiderstand: & \strWidMittelwert  & \strWidStandardabweichung \\
		Spezifischer Str�mungswiderstand: & \spezStrWidMittelwert & \spezStrWidStandardabweichung\\ 
		L�ngenbezogener Str�mungswiderstand. & \laengenbezStrWidMittelwert & \laengenbezStrWidStandardabweichung \\
	&&\\
 	\hline 
\end{tabular}


\begin{itemize}
	\item der hub
	\item Fl�che und Abmessungen  der Probe, 
	\item befestigung der probe
	\item DIN oder besser ISO?
	\item Gewicht kann man weglassen, wa?
	\item Frequenz von 2 Hz
	\item Layout; mehr abst�nde einbauen
	\item std ? s = ? 
\end{itemize}

\end{document}